%========================================================
\section{Martin-\Lof's Dependent Type Theory: A Topological Lens}\label{sec:mltt}

Modern homotopy type theory rests on three structural features of
Martin-\Lof\ dependent type theory—\emph{universes},
\emph{dependent types}, and \emph{identity types}.  Interpreting each
through a geometric lens reveals how the syntax of proofs encodes
classical homotopy theoretic ideas.

\subsection{Universes and Type Families as Stratified Spaces}
To quantify over types without reproducing Russell-style paradoxes
Martin-\Lof\ introduced an ascending hierarchy
\[
    \U_{0} : \U_{1} : \U_{2} : \cdots,
\]
where every level \(\U_{i}\) itself lives in the next, a property
called \emph{cumulativity}.
We can notice the same pattern when we are assembling a CW-complex:
the \(n\)-skeleton depends only on cells attached in lower dimensions,
yet remains a subspace of the completed object.
Thus each universe operates like a skeleton that supports all types—and hence all spaces—constructed so far,
while leaving room to adjoin further ``cells'' at the next stage.

Given a base type \(A\) a \emph{type family} \(B:A\to\U\) assigns to
every point \(x:A\) a fibre \(B(x)\).  Topologically one recognises
the assignment as a fibration \(p:\Sum_{x:A}B(x)\to A\) whose
total space is the \(\Sigma\)-type \(\Sum_{x:A}B(x)\) and whose fibre
over \(x\) is literally \(B(x)\).  When \(B\) is the constant function
\(\lambda(x:A).~C\) the fibration becomes trivial,
\(A\times C\xrightarrow{\;\pi_{1}\;}A\); when \(B\) varies non-trivially
one obtains twisted bundles, covering spaces, or more exotic
constructions such as local systems of higher groupoids.

\subsection{Dependent Types as Continuous Constructions}
\begin{definition}[$\Pi$-types as Sections]\label{def:dep-func-type}
    Given a family \(B:A\to\U\) the dependent function
    type \(\Prod_{(x:A)}B(x)\) consists of terms that pick a point
    \(f(x):B(x)\) \emph{continuously} in \(x\).  Equivalently a term
    \(f\) determines a section \(s:A\to\Sum_{x:A}B(x)\) via
    \(s(x)\DefEq(x,f(x))\), and the definitional equality
    \(\pi\circ s=\Id_{A}\) reads ``\(s\) is a genuine section'' after
    projection.
\end{definition}

From a logical standpoint \(\Pi\)-types internalize universal
quantification: the judgment \(t:\Prod_{x:A}B(x)\) is a constructive
proof of ``for every \(x\) we have \(B(x)\)''.  From a homotopical
standpoint sections detect whether a bundle admits global trivializing
data, a theme that resurfaces when \(\Pi\)-types are used to model
parallel transport and connection forms.

\begin{definition}[$\Sigma$-types as Total Spaces]\label{def:dep-pair-type}
    Dually, the dependent pair type \(\Sum_{(x:A)}B(x)\) represents the
    space obtained by \emph{gluing} all fibres together.  A point
    \((a,b):\Sum_{x:A}B(x)\) simultaneously witnesses the existence of
    \(a:A\) and a dependent element \(b:B(a)\); it therefore embodies the
    constructive content of the existential quantifier.  In topology the
    same ordered pair marks a geometric point lying over \(a\) in the
    total space of the fibration.
\end{definition}

Because \(\Sigma\) and \(\Pi\) are adjoint
(yes, by this we mean adjunctions in category theory)
in an appropriate sense,
one already glimpses a categorical formulation of fibre bundles:
sections are right adjoints to projection, while total-space formation
is left adjoint.

\subsection{Identity Types as Path Spaces}
The conceptual breakthrough of HoTT is to interpret propositional equality as a space of paths in the topological sense.
For two terms \(a,b:A\) the identity type \(\Id_{A}(a,b)\) carries as inhabitants the paths from \(a\) to \(b\);
the canonical constructor \(\Refl_{a}:\Id_{A}(a,a)\) is the constant path.  
Two radically different equalities coexist:

\begin{table}[htbp]
    \begin{tabularx}
        {\textwidth}
        {>{\raggedright\arraybackslash}p{0.5\textwidth}>{\raggedright\arraybackslash}p{0.5\textwidth}}
        \toprule
        \(\displaystyle a\equiv b:A\)                    & \(\displaystyle p:\Id_{A}(a,b)\) \\
        \midrule
        judgmental, enforced by definitional computation &
        propositional, a geometric path that may vary                                       \\
        \bottomrule
    \end{tabularx}
\end{table}

Judgmental (or definitional) equality corresponds to syntactic normalization, 
whereas propositional equality admits many distinct witnesses; 
the latter is where homotopy becomes useful.

\subsection{Path Induction - Contracting the Interval}
Every identity type is governed by the \emph{path induction} (or
\(\mathsf{J}\)) rule: to prove a property \(C\) of arbitrary paths it
suffices to prove it for the trivial path.  Formally, given a family
\(C:\Prod_{x,y:A}\Id_{A}(x,y)\to\U\) and a term
\(d:\Prod_{x:A}C(x,x,\Refl_{x})\) there exists a function
\[
    \mathsf{J}(d):\Prod_{x,y:A}\Prod_{p:\Id_{A}(x,y)}C(x,y,p),
\]
unique up to definitional equality.  From a geometric vantage point
path induction says that the unit interval contracts onto a point, so
any construction that is homotopy-invariant over the interval already
follows from its value at the constant path.

\subsection{A First Glimpse of Homotopy}\label{subsec:first-homotopy}
Suppose \(p,q:\Id_{A}(a,b)\) are two non-identical proofs of equality.
They materialize as distinct paths between \(a\) and \(b\).
Applying identity types once more,
\(\Id_{\Id_{A}(a,b)}(p,q)\) is the type of homotopies \(p\Rightarrow q\),
and iterating the process generates a tower of higher paths whose
globular structure equips every type with an \(\Inf\)-groupoid of its
points, paths, homotopies, and so on.  Grothendieck envisioned such
structures as the language in which algebraic topology should be
carried out; in HoTT they arise automatically from the basic rules of
type formation and path induction.

So far, we have now reinterpreted the building blocks of Martin-\Lof\ theory — universes, dependent types, identity types, and path induction --- through geometry.  
Table~\ref{tab:hott-dict} summarizes the interpretation of Type Theory in Logic and Topology.