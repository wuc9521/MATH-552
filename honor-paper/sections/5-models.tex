\section{Semantic Models at a Glance}\label{sec:models}

The consistency of HoTT rests on mathematical models that bridge its syntax with concrete geometric or computational structures. 
Two models stand out for their philosophical and practical implications.  

The Kan Simplicial Set Model~\cite{kan1955abstract} interprets types as spaces built from simplices—higher-dimensional analogs of triangles and tetrahedrons. This model validates univalence by treating type equivalences as homotopy equivalences. However, it lacks computational rules for paths: while it guarantees the existence of a path \(a = b\), it does not specify how to construct or compute it.  

A more constructive approach is offered by Cubical Type Theory~\cite{cohen2016cubicaltypetheoryconstructive}. 
This model introduces an \emph{interval object} \(\mathbb{I}\) (representing a continuum from 0 to 1) and \emph{connection operations} to glue higher-dimensional cubes. 
Here, paths become computable functions over \(\mathbb{I}\). For instance, the loop \(\textsf{loop} : S^1\) in Cubical Agda can be evaluated as a concrete sequence of interval manipulations. This model not only justifies univalence constructively but also enables algorithms for normalizing higher-dimensional paths, as demonstrated in our analysis of \(\pi_1(S^1)\) in Section~\ref{sec:components}.  

These models are not mere abstractions. They ground HoTT's synthetic reasoning in computation: when we assert \(\pi_1(S^1) \simeq \mathbb{Z}\), the cubical model ensures this equivalence is not just symbolic—it reduces to executable code manipulating integers. This synergy between syntax and semantics positions HoTT as both a foundational theory and a practical tool for formal mathematics.  