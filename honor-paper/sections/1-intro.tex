\section{Why Homotopy inside Type Theory?}\label{sec:intro}

Homotopy Type Theory (HoTT) merges two seemingly disparate fields: the logical rigor of type theory and the geometric intuition of homotopy theory. To appreciate its significance, we begin by contextualizing its foundations.

\emph{Type Theory} formalizes mathematics through a computational lens. In this framework, propositions are represented as \emph{types}, and proofs as \emph{terms} inhabiting those types. For instance, the statement ``2 is even'' corresponds to a type \(\textsf{Even}(2)\), and a proof of this statement is a term \(t : \textsf{Even}(2)\). 
\emph{Dependent Type Theory} extends this idea by allowing types to depend on terms: if \(A\) is a type and \(B(x)\) is a type for each \(x : A\), the dependent product type \(\Prod_{x:A} B(x)\) encodes universal quantification (``for all \(x : A\), \(B(x)\) holds''). 
This mechanism underpins modern proof assistants like Coq~\cite{bauer2016hottlibraryformalizationhomotopy} and Agda~\cite{Agda}. 
However, traditional type theory treats equality simplistically—terms \(a\) and \(b\) are either judgmentally equal (\(a \DefEq b\)) or not, discarding any higher-dimensional structure behind why they are equal.

\emph{Homotopy Theory}, on the other hand, studies spaces through continuous deformations: paths between points, homotopies between paths, and so on. Vladimir Voevodsky’s groundbreaking insight was to reinterpret types as spaces, terms as points, and equality proofs \(p : a = b\) as paths from \(a\) to \(b\).
HoTT elevates equality to a first-class geometric notion, preserving not just whether two terms are equal but how they are equal.
For example, two programs proven equal in HoTT may follow distinct computational paths, analogous to different routes connecting the same endpoints.

The fusion of these ideas crystallized with Hofmann and Streicher’s 1994 groupoid model and Voevodsky’s later contributions: the \emph{univalence axiom} (equating equivalent types) and \emph{higher inductive types}~\cite{hottbook} (defining spaces by specifying points and paths). Together, they enable HoTT to internalize homotopy-theoretic concepts directly into type theory. For computer scientists, this enriches program verification with geometric reasoning; for mathematicians, it offers a language where proofs are both human-readable and machine-checkable.