\noindent
In this paper I try to answer the question: \emph{what makes Homotopy Type Theory (HoTT) genuinely homotopic?} 
By bridging Martin-\Lof's dependent type theory and classical homotopy theory, 
we (1) show how Martin-\Lof type theory's core components - universes, dependent types, and identity types - naturally encode geometric concepts when viewed through a homotopical lens, 
(2) demonstrate how HoTT internalizes five key homotopy-theoretic ideas: paths as equality proofs, fibrations as dependent types, equivalences, univalence, and higher inductive types, and 
(3) apply these to reconstruct two classical results: the computation of $\pi_1(S^1) \simeq \mathbb{Z}$ via path induction and transport, and the Seifert-van Kampen theorem through pushouts of higher inductive types. 
The analysis reveals how HoTT captures topological phenomena synthetically, replacing analytic machinery with type-theoretic primitives.