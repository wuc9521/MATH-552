%========================================================
\section{Key Homotopic Ingredients of HoTT}\label{sec:components}

In this section we are going to analyze five important \emph{synthetic}\footnote{``Synthetic'' (in HoTT) refers to reasoning about spaces and paths directly within type theory without explicitly constructing their underlying topological or set-theoretic models.} ingredients of homotopy in HoTT.
While there are many more advanced aspects to this topic, to the best of my knowledge from MATH~552, these are the key concepts I've been able to properly understand (given time constraints).

\subsection{Paths as Homotopies and Covering Spaces}\label{sec:paths}
Given a type \(A\) and points \(a,b:A\), the identity type
\(\Paths_A(a,b)\DefEq\Id_A(a,b)\) is interpreted homotopically as the
space of paths connecting \(a\) to \(b\).
Repeating the identity-type construction endows every type with an entire
tower of higher path spaces \(\Paths_A^{n}(a,b)\), and the coherence laws of
identity establish an \(\Inf\)-groupoid structure.
In classical language this means that reflexivity, symmetry, transitivity,
and their higher analogues are not separate axioms but canonical
operations internal to \(\U\).

A basic but instructive illustration is the circle \(S^1\).  
Presenting \(S^1\) as a higher inductive type with a single point constructor \(\Base\) and a loop constructor
\(\Loop:\Paths_{S^1}(\Base,\Base)\) already forces the entire set of
homotopy classes of maps out of the circle to satisfy the desired
universal property; no appeal to piecewise linear charts or CW
structures is necessary.  In this way HoTT replaces analytic path
concatenation by the purely syntactic operation of transitivity in
\(\Id\)-types.

A dependent type \(P:A\to\U\) then behaves as a fibration whose total
space is \(\Sum_{x:A}P(x)\).  The canonical projection
\(\pi:\Sum_{x:A}P(x)\to A\) mirrors a covering map in topology.  The
\emph{path induction} principle asserts that to define data depending
on an arbitrary path it suffices to specify the case of the
trivial path \(\Refl_a\); this ``contracts the interval'' and yields the
synthetic counterpart of the homotopy lifting property.  The induced
transport map
\[
  \Transport^{P}_{q}:P(b_0)\longrightarrow P(b_1)\qquad(q:\Paths_B(b_0,b_1))
\]
is formally unique and automatically functorial, thus reproducing the
monodromy of a covering space without explicit reference to
topological microstructure.

\subsection{\texorpdfstring{$\Pi$}{Π}-Types as Sections of Fibrations}\label{sec:fibrations}
Where topology studies continuous sections of a fibration
\(\pi:E\to A\), type theory studies terms of the dependent function
type \(\Prod_{x:A}P(x)\).  A term
\(f:\Prod_{x:A}P(x)\) canonically determines a section
\(s:A\to\Sum_{x:A}P(x)\) by the rule \(s(x)\DefEq(x,f(x))\), and the
typing judgement \(\mathsf{ap}_{\pi}(f(x))=\Refl_x\) is literally the
equation \(\pi\circ s=\Id_A\) transcribed into the internal language.
Consequently the classical slogan ``sections are homotopies'' acquires a
formal meaning: \(\Pi\)-types serve simultaneously as spaces of
global elements and as witnesses of path-based coherence.

\begin{definition}
  A dependent map \(f:\Prod_{x:A}P(x)\) is called \emph{constant up to
    homotopy} if for all \(x,y:A\) there exists a path
  \(p:\Paths_{P(y)}(f(x\mathrel{=}y),f(y))\) whose projection onto \(A\)
  is \(\Refl_y\).  Constant maps in this sense correspond to \(\pi_1\)-invariant
  sections of a covering space in the classical setting.
\end{definition}

This perspective clarifies why \(\Pi\)-types satisfy functional
extensionality whenever univalence holds: two sections are equal just
when they are pointwise connected by a path, precisely the condition
under which they represent the same global section.

\subsection{Equivalences and the Universal Cover}\label{sec:univalence}
A map \(f:A\to B\) is an \emph{equivalence} if each of its fibres
\(\Sum_{b:B}\Paths_B(f(a),b)\) is contractible.  Univalence enhances
this notion by stipulating that the type \((A\simeq B)\) of
equivalences coincides with the identity type \((A=B)\).  Equivalences
are therefore paths in \(\U\), so that ``moving along'' a family of
equivalences simply is transport in \(\U\).

\begin{theorem}
  \label{thm:omegaS1}
  The loop space of the circle satisfies \(\Omega S^1\simeq\Z\).
\end{theorem}

\begin{proof}[Sketch]
  Define \(\Code:S^1\to\U\) by \(\Code(\Base)\DefEq\Z\) and
  \(\Transport^{\Code}(\Loop)\DefEq\Suc\).  The total space
  \(\Sum_{x:S^1}\Code(x)\) carries a projection with homotopy-lifting
  behaviour identical to the exponential map \(\exp:\R\to S^1\).
  Contractibility of every fibre implies that paths at \(\Base\) are in
  bijective correspondence with integers, yielding the stated
  equivalence without leaving type theory.
\end{proof}

Because the argument never mentions open covers or local triviality,
it demonstrates that HoTT faithfully reproduces covering-space
phenomena using purely synthetic tools.

\subsection{Lifting Properties and Transport}\label{sec:lifting}
In classical algebraic topology the homotopy lifting property (HLP)
says that a covering map \(p:E\to B\) allows every homotopy
\(H:X\times[0,1]\to B\) to be lifted uniquely once its restriction to
\(X\times\{0\}\) is fixed.  HoTT wraps the entire content of HLP into
the functoriality and higher coherences of \(\Transport\).

Let \(q:\Paths_B(b_0,b_1)\) and \(e_0:P(b_0)\).
The dependent elimination rule for identity types produces the unique term \(\Transport^{P}_{q}(e_0):P(b_1)\).
Naturality of \(\Transport\) under path concatenation encodes the uniqueness clause of path lifting, 
while the computation rule for \(\Refl_{b_0}\) encodes the starting-point condition.
Higher paths between paths in \(B\) lift to dependent paths between the
corresponding transport maps, providing the synthetic substitute for
homotopy lifting.
Hence the entire monodromy representation
\(\pi_1(B)\to\mathsf{Aut}(P(b_0))\) can be recovered from first
principles of dependent type theory.

\subsection{Free Groups and the Van Kampen Theorem}\label{sec:vankampen}
The Seifert-Van Kampen theorem (SVK) computes the fundamental group of
a union from the fundamental groups of its parts and their
intersection,
\[
  \pi_1(U\cup V)\simeq\pi_1(U)*_{\pi_1(U\cap V)}\pi_1(V).
\]
HoTT reconstructs SVK synthetically via pushouts of higher inductive
types.  Suppose \(X\) is the pushout of maps \(i:W\to U\) and
\(j:W\to V\).  Because pushouts in HoTT satisfy the same universal
property as in homotopy theory, mapping out of \(X\) into any other
type induces an equivalence between dependent function spaces
mirroring the amalgamated free product above.

\begin{example}[Wedge of two circles]
  Present \(S^1\vee S^1\) by two point constructors
  \(\Base_1,\Base_2\) and two loop constructors \(\Loop_1,\Loop_2\).
  Any map from \(S^1\vee S^1\) into a group-valued Eilenberg–MacLane
  space is completely specified by the images of \(\Loop_1\) and
  \(\Loop_2\), and the concatenation law for loops translates into the
  group law in the free product \(\Z*\Z\).  Path induction formalizes
  this reasoning, yielding a direct proof of
  \(\pi_1(S^1\vee S^1)\simeq\Z*\Z\) internal to HoTT.
\end{example}

The synthetic route not only avoids piecewise-linear approximations but
also produces computational content: the proof yields an explicit
algorithm extracting normal forms of words in \(\pi_1(S^1\vee S^1)\).

\subsection{Fundamental Group Action via Deck Transformations}\label{sec:deck}
Let \(p:\tilde X\to X\) be a universal cover.  The group
\(\Deck(\tilde X)\) of deck transformations consists of those
self-equivalences of \(\tilde X\) that commute with \(p\).
In HoTT the covering is a dependent type
\(\Code:X\to\U\); its total space
\(\Sum_{x:X}\Code(x)\) inherits an internal \(\Deck\)-action from
transport.

\begin{theorem}
  For \(X=S^1\) and \(\Code\) as in Theorem~\ref{thm:omegaS1},
  \(\Deck\bigl(\Sum_{x:S^1}\Code(x)\bigr)\simeq\Z\).
\end{theorem}

\begin{proof}[Idea]
  Any integer \(n:\Z\) gives an equivalence
  \(f_n:(x,k)\mapsto(x,k+n)\), and
  \(\Transport^{\Code}(\Loop^{n})=\lambda k.~k+n\) shows that these
  exhaust all equivalences commuting with the projection.  Functoriality
  of \(\Transport\) yields a group homomorphism \(\Z\to\Deck\) that is
  easily seen to be inverse to the map sending a deck transformation to
  its action on the distinguished fibre over \(\Base\).
\end{proof}

Because deck transformations are equivalences in the sense of
univalence, their classification by \(\Z\) is witnessed by a path in
\(\U\), once again turning a geometric identification into an internal
equality.

\subsection{Synthesis: HoTT as Synthetic Homotopy Theory}\label{sec:synthesis}
Collecting the preceding themes, one may characterise HoTT as a
framework where traditional homotopy theory is recovered from the
interaction of four primitive notions:

\medskip
\noindent
\emph{(i) identity types} model paths and higher paths;
\emph{(ii) dependent types} encode fibrations and covering spaces;
\emph{(iii) higher inductive types} allow defining spaces by specifying both points and paths (e.g., attaching cells or gluing components), which directly encodes theorems like Seifert-van Kampen within type theory.
\emph{(iv) univalence} unifies equivalence with identity, letting group
actions and deck transformations be expressed by transport.

\medskip
\noindent
Because these ingredients are computationally meaningful, classical
homotopical arguments admit direct formalization in proof assistants
such as \textsc{Agda}~\cite{Agda}, \textsc{Coq-HoTT}~\cite{bauer2016hottlibraryformalizationhomotopy}, or \textsc{Lean}~\cite{10.1007/978-3-030-79876-5_37}.  The cost
of analytic overhead is thus exchanged for the benefit of executable
proofs whose correctness rests solely on the constructive core of
type theory.
