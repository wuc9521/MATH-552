%========================================================
\section{Key Homotopic Ingredients of HoTT}\label{sec:components}

In this section we will introduce some classical ``\emph{synthetic}''\footnote{In HoTT, ``synthetic'' refers to reasoning about spaces and paths directly within type theory without relying on external topological models.} homotopic concepts internalized into HoTT.
While there are many more advanced aspects to this topic, to the best of my knowledge from MATH~552, these are the key concepts I've been able to properly understand (given time constraints).

\begin{definition}[Contractible type]\label{def:contract}
  A type \(C\) is \emph{contractible} if there exists a point \(c_0:C\) along with a path
  \(\Prod_{c:C}\Id_C(c,c_0)\). Geometrically, \(C\) resembles a ``one-point space'' with only higher-order redundancies, and is obviously corresponded with contractible spaces in homotopy.
\end{definition}

\begin{definition}[Equivalence]\label{def:equiv}
  A map \(f:A\to B\) is an \emph{equivalence} when each fiber
  \(\Sum_{b:B}\Id_B\Paren{f(a),b}\) is contractible. Logically, this means
  ``\(f\) is both surjective and injective'',  while from a homotopical perspective, it is called a weak homotopy equivalence.
\end{definition}

\begin{definition}[Higher-inductive type (HIT)]\label{def:hit}
  HITs extend ordinary inductive types: in addition to point constructors, they allow \emph{path constructors} (and even higher paths)
  to explicitly generate specified equalities. For example, the circle
  \(S^1\) is defined by two constructors:
  \(\Base:S^1\) and \(\Loop:\Id_{S^1}(\Base,\Base)\).
\end{definition}

\begin{definition}[Pushout]\label{def:pushout}
  Given \(W\xrightarrow{i}U,\,W\xrightarrow{j}V\), their \emph{pushout} is an HIT
  with inclusion constructors \( \iota_U:U\to\mathsf{PO}\) and \(\iota_V:V\to\mathsf{PO}\),
  plus a path constructor
  \( \alpha:\Id_{\mathsf{PO}}\!\Paren{\iota_U\!\circ i(w),\,\iota_V\!\circ j(w)}\).
  This universal property aligns perfectly with topological pushouts in HoTT.
\end{definition}

\begin{definition}[Eilenberg–MacLane space]\label{def:EM}
  For an abelian group \(G\) and integer \(n\ge1\), \(K(G,n)\) is the unique connected space with homotopy group
  \(\pi_n K(G,n)\cong G\) concentrated in degree \(n\). HoTT can represent \(K(G,1)\) using HITs,
  making its loop space exactly \(G\), with all higher paths trivialized.
\end{definition}

%--------------------------------------------------------
\subsection{Paths as Homotopies and Covering Spaces}\label{sec:paths}
Given a type \(A\) and points \(a,b:A\), 
the identity type \(\Paths_A(a,b)\DefEq\Id_A(a,b)\) is homotopically interpreted as the path space connecting \(a\) and \(b\).
Iterating identity types yields higher-order paths \(\Paths^{n}_A(a,b)\), 
whose built-in reflexivity, symmetry, and transitivity laws generate an \(\Inf\)-groupoid structure—realizing Grothendieck's vision of \(\Inf\)-groupoids.

The HIT presentation of the circle \(S^1\) -- \(\Base\) and \(\Loop\) -- ensures that any map from \(S^1\) automatically satisfies the desired
``unique recursion centered at the base point'' universal property, thereby replacing analytic constructions like CW-structures or piecewise glueing with purely algebraic methods.

For a dependent type \(P:A\to\U\), its total space
\(\Sum_{x:A}P(x)\) with projection
\(\pi\) forms a \emph{fibration}.
\emph{Path induction} (i.e., the \(J\)-rule) asserts: to define data dependent on arbitrary paths, it suffices to assign values for the trivial path
\(\Refl_a\). Geometrically, this corresponds to interval contraction, producing a unique and functorial
\[
  \Transport^{P}_{q}:P(b_0)\to P(b_1)\qquad(q:\Paths_A(b_0,b_1)),
\]
thereby reconstructing the monodromy of covering spaces without explicit ``lifting.''

%--------------------------------------------------------
\subsection{\texorpdfstring{$\Pi$}{Π}-Types as Sections of Fibrations}\label{sec:fibrations}
Topology studies continuous sections of a fibration \(\pi:E\to A\); type theory's counterpart is the dependent function type
\(\Prod_{x:A}P(x)\).
For \(f\) inhabiting this type,
\(s(x)\DefEq(x,f(x))\) defines a section
\(s:A\to\Sum_{x:A}P(x)\),
while the judgment
\(\mathsf{ap}_{\pi}(f(x))=\Refl_x\)
directly translates \(\pi\circ s=\Id_A\) into the internal language.

\begin{definition}
  If there exists a path
  \(\Prod_{x,y:A}\Paths_{P(y)}\!\Paren{\,f(x\mathrel{=}y),\,f(y)}\)
  whose projection onto \(A\) is \(\Refl_y\), then \(f\) is called \emph{constant up to homotopy}. This is equivalent to \(\pi_1(A)\)-invariant sections in the classical sense.
\end{definition}

Under Voevodsky's \emph{univalence}, pointwise connectedness of two sections implies
\emph{judgmental equality}, providing an intrinsic source for function extensionality in HoTT.

%--------------------------------------------------------
\subsection{Equivalences and the Universal Cover}\label{sec:univalence}
By Definition \ref{def:equiv}, a map \(f:A\to B\) is an equivalence if all its fibers are contractible.
The \emph{univalence axiom} further asserts
\((A\simeq B)\simeq(A=B)\), elevating equivalences to paths in the universe.

\begin{theorem}\label{thm:omegaS1}
  The loop space \(\Omega S^1\) is equivalent to \(\Z\): \(\Omega S^1\simeq\Z\).
\end{theorem}

\begin{proof}[Sketch]
  Define \(\Code:S^1\to\U\) with
  \(\Code(\Base)\DefEq\Z\) and
  \(\Transport^{\Code}(\Loop)\DefEq\Suc\).
  The projection of the total space
  \(\Sum_{x:S^1}\Code(x)\) onto \(S^1\) behaves identically to the classical exponential map
  \(\exp:\R\to S^1\).
  Contractibility of each fiber implies a bijection between
  \(\Paths_{S^1}(\Base,\Base)\) and \(\Z\).
\end{proof}

Without invoking open covers or local trivializations, this example demonstrates how HoTT synthetically reconstructs covering space theory.

%--------------------------------------------------------
\subsection{Lifting Properties and Transport}\label{sec:lifting}
Classical HLP asserts: a covering \(p:E\to B\) uniquely lifts any homotopy
\(H:X\times I\to B\). In HoTT,
the functoriality and higher coherence of
\(\Transport\) perfectly encapsulate this uniqueness and existence:
Given \(q:\Paths_B(b_0,b_1)\) and \(e_0:P(b_0)\),
we necessarily obtain a unique
\(\Transport^{P}_{q}(e_0):P(b_1)\),
with path concatenation corresponding to function composition.
Consequently, the entire monodromy action
\(\pi_1(B)\to\mathsf{Aut}(P(b_0))\)
is directly derived from type-theoretic principles.

%--------------------------------------------------------
\subsection{Free Groups and the Seifert–van Kampen Theorem}\label{sec:vankampen}
SvK computes the fundamental group of a union:
\[
  \pi_1(U\cup V)\simeq
  \pi_1(U)*_{\pi_1(U\cap V)}\pi_1(V).
\]
HoTT reconstructs this via the pushout HIT from Definition \ref{def:pushout}.
Let
\(\mathsf{PO}\) be the pushout of
\(W\xrightarrow{i}U,\,W\xrightarrow{j}V\);
then for any \(Z\),
\[
  \Bigl(\Prod_{\mathsf{PO}} Z\Bigr)\;\simeq\;
  \Bigl(\Prod_U Z\Bigr)\;\times_{\Prod_W Z}\;
  \Bigl(\Prod_V Z\Bigr),
\]
where this isomorphism replaces group amalgamation at the section level.

\begin{example}[\(S^1\vee S^1\)]
  Generated by point constructors \(\Base_1,\Base_2\) and path constructors
  \(\Loop_1,\Loop_2\),
  maps into any \(K(G,1)\) (Definition \ref{def:EM}) depend solely on the images of
  \(\Loop_1,\Loop_2\), yielding
  \(\pi_1(S^1\vee S^1)\simeq\Z*\Z\).
\end{example}

This ``merge-as-pushout'' proof can output normal forms via compilers, providing algorithmically meaningful free product reduction.

%--------------------------------------------------------
\subsection{Fundamental Group Action via Deck Transformations}\label{sec:deck}
In the covering \(\tilde X\xrightarrow{p}X\),
\(\Deck(\tilde X)\) is the group of all self-equivalences commuting with \(p\).
Encoding the covering as a dependent type
\(\Code:X\to\U\),
\(\Deck\bigl(\Sum_{x:X}\Code(x)\bigr)\)
precisely comprises equivalences preserving the projection.

\begin{theorem}\label{thm:deckZ}
  For \(X=S^1\) and \(\Code\) as in Theorem \ref{thm:omegaS1},
  \(\Deck\bigl(\Sum_{x:S^1}\Code(x)\bigr)\simeq\Z\).
\end{theorem}

\begin{proof}[Idea]
  Each integer \(n:\Z\) generates an equivalence
  \(f_n:(x,k)\mapsto(x,k+n)\),
  while
  \(\Transport^{\Code}(\Loop^{\,n})=\lambda k.\,k+n\)
  shows no other equivalences satisfy the commutativity condition.
\end{proof}

Since deck transformations are themselves equivalences,
univalence promotes the ``\(\simeq\)'' here to a path in \(\U\).
