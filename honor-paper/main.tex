\documentclass[12pt]{amsart}
\usepackage[T1]{fontenc}
\usepackage[margin=1in]{geometry} 
\usepackage{setspace}             
\usepackage[T1]{fontenc}
\usepackage[margin=1in]{geometry} 
\usepackage{setspace}             
\usepackage{amsmath,amssymb,amsthm,newtxmath}
\usepackage{mathrsfs}
\usepackage{indentfirst}
\usepackage[utf8]{inputenc}
\usepackage{pifont}
\usepackage{lastpage}
\usepackage{hyperref}
\usepackage{cleveref}
\usepackage{float}
\usepackage{array}
\usepackage{mathtools}
\usepackage{tikz, tikz-cd}
\usepackage{booktabs}
\usepackage{tabularx}
\usepackage{longtable} % for breakable table
\usepackage{mathpartir}
\newtheorem{theorem}[subsection]{Theorem}
\newcommand{\Lof}{L\"{o}f}
\newcommand{\Inf}{\infty}
\newcommand{\Defeq}{:\equiv}
\newcommand{\Paren}[1]{\left\lparen #1 \right\rparen}
\newcommand{\Sum}{\textstyle\mathbf{\sum}}
\newcommand{\Prod}{\textstyle\mathbf{\prod}}
\theoremstyle{definition}
\newtheorem{definition}{Definition}[section]

\title{What is homotopical about Homotopy Type Theory?}
\author{Chentian Wu}
\date{}

\begin{document}
\maketitle

\begin{abstract}
  This paper explores the profound connection between homotopy theory and dependent type theory that gave rise to Homotopy Type Theory (HoTT~\cite{hottbook}). We investigate how the homotopical interpretation transforms our understanding of type theory, where types are viewed not merely as sets but as topological spaces up to homotopy equivalence. By examining the unexpected isomorphism between Martin-\Lof's identity types and path spaces, we reveal how this correspondence naturally led to the univalence axiom—a principle with no precedent in classical foundations. We analyze specific constructions in HoTT that directly mirror algebraic topology concepts, including higher inductive types as synthetic representations of cell complexes. This exploration illuminates why homotopy—rather than other mathematical structures—offered the perfect framework for resolving long-standing issues in intensional type theory while simultaneously providing new foundations for mathematics that naturally accommodate higher-dimensional structures.
\end{abstract}

\tableofcontents
\section{Introduction to Type Theory}\label{sec:tt}

% HoTT 在数学中是什么地位


\section{The Homotopical Structure of Type Theory}\label{sec:hott}

In this section we give the basic idea of HoTT 

\begin{enumerate}
    \item Martin-\Lof's identity types as path spaces
    \item The Hofmann-Streicher groupoid interpretation
    \item n-types and truncation: from sets to ∞-groupoids
    \item Transport and path induction as homotopical operations
\end{enumerate}
\section{Univalence: The Core Homotopical Axiom}\label{sec:univalence}

In this section we mainly talk about the univalence axiom in HoTT.

\begin{enumerate}
    \item Informal meaning: equivalent types are equal
    \item Topological motivation: homotopy equivalence and its role
    \item Consequences of univalence for mathematics
    \item Comparison with traditional foundations
\end{enumerate}



\section{Higher Inductive Types: Synthetic Topology}\label{sec:hit}

\begin{enumerate}
    \item Circle $S^1$, spheres $S^n$, and torus $T^2$ as HITs
    \item Computing fundamental groups synthetically
    \item Comparison with classical topological constructions
    \item Eilenberg-MacLane spaces and cohomology in HoTT
\end{enumerate}

\section{Computational Aspects and Cubical Models}\label{sec:comp}

\begin{enumerate}
    \item Challenges of computational interpretation
    \item Cubical type theory: paths as functions from the interval
    \item Providing computational content to homotopical concepts
    \item Implementation in proof assistants
\end{enumerate}


\section{Why Homotopy? The Unique Fit}\label{sec:answer}

\begin{enumerate}
    \item What makes the homotopy interpretation distinctly powerful
    \item Alternative interpretations and their limitations
    \item How homotopy addresses longstanding type-theoretical problems
    \item Connections to higher category theory
\end{enumerate}


\section{Conclusion and Future Directions}\label{sec:conc}

\begin{enumerate}
    \item The essential homotopical nature of HoTT
    \item Applications in mathematics and computer science
    \item Open questions and research opportunities
    \item The future of homotopical foundations
\end{enumerate}


\bibliographystyle{amsplain}
\bibliography{refs}

\end{document}