\documentclass[12pt]{amsart}
\usepackage[T1]{fontenc}
\usepackage[margin=1in]{geometry} 
\usepackage{setspace}             
\usepackage[T1]{fontenc}
\usepackage{geometry}
\usepackage{amsmath,amssymb,amsthm}
\usepackage{mathrsfs}
\usepackage{indentfirst}
\usepackage[utf8]{inputenc}
\usepackage{lastpage}
\usepackage{hyperref}
\usepackage{cleveref}
\usepackage{array}
\usepackage{float}
\usepackage{mathtools}
\usepackage{tikz, tikz-cd}
\newtheorem{theorem}[subsection]{Theorem}
\newcommand{\Lof}{L\"{o}f}

\title{What is homotopical about Homotopy Type Theory?}
\author{Chentian Wu}
\date{}

\begin{document}
\maketitle

\begin{abstract}
  % This paper explores the profound connection between homotopy theory and dependent type theory that gave rise to Homotopy Type Theory (HoTT~\cite{hottbook}). We investigate how the homotopical interpretation transforms our understanding of type theory, where types are viewed not merely as sets but as topological spaces up to homotopy equivalence. By examining the unexpected isomorphism between Martin-\Lof's identity types and path spaces, we reveal how this correspondence naturally led to the univalence axiom—a principle with no precedent in classical foundations. We analyze specific constructions in HoTT that directly mirror algebraic topology concepts, including higher inductive types as synthetic representations of cell complexes. This exploration illuminates why homotopy—rather than other mathematical structures—offered the perfect framework for resolving long-standing issues in intensional type theory while simultaneously providing new foundations for mathematics that naturally accommodate higher-dimensional structures.
  The introduction of homotopy theory into type theory was firsyly mentioned in the  1994 paper~\cite{hofmann1994groupoid} of Martin Hofmann and Thomas Streicher.
  In this paper, I'll give a brief answer to the question: ``What's homotopical about Homotopy Type Theory''.
  I'll firstly give a introduction of Martin-\Lof's dependent type theory, which is the mathematical background (or object) that we care about in the context of HoTT.
  After that, I'll give a correspondence of common concepts in homotopy theory and HoTT.
  % Homotopy type theory is (among other things) a foundational language for mathematics, i.e., an alternative to Zermelo–Fraenkel set theory.
\end{abstract}

\tableofcontents
\section{Why Homotopy inside Type Theory?}\label{sec:intro}

Homotopy Type Theory (HoTT) merges two seemingly disparate fields: the logical rigor of type theory and the geometric intuition of homotopy theory. To appreciate its significance, we begin by contextualizing its foundations.

\emph{Type Theory} formalizes mathematics through a computational lens. In this framework, propositions are represented as \emph{types}, and proofs as \emph{terms} inhabiting those types. For instance, the statement ``2 is even'' corresponds to a type \(\textsf{Even}(2)\), and a proof of this statement is a term \(t : \textsf{Even}(2)\). 
\emph{Dependent Type Theory} extends this idea by allowing types to depend on terms: if \(A\) is a type and \(B(x)\) is a type for each \(x : A\), the dependent product type \(\Prod_{x:A} B(x)\) encodes universal quantification (``for all \(x : A\), \(B(x)\) holds''). 
This mechanism underpins modern proof assistants like Coq~\cite{bauer2016hottlibraryformalizationhomotopy} and Agda~\cite{Agda}. 
However, traditional type theory treats equality simplistically—terms \(a\) and \(b\) are either judgmentally equal (\(a \DefEq b\)) or not, discarding any higher-dimensional structure behind why they are equal.

\emph{Homotopy Theory}, on the other hand, studies spaces through continuous deformations: paths between points, homotopies between paths, and so on. Vladimir Voevodsky’s groundbreaking insight was to reinterpret types as spaces, terms as points, and equality proofs \(p : a = b\) as paths from \(a\) to \(b\).
HoTT elevates equality to a first-class geometric notion, preserving not just whether two terms are equal but how they are equal.
For example, two programs proven equal in HoTT may follow distinct computational paths, analogous to different routes connecting the same endpoints.

The fusion of these ideas crystallized with Hofmann and Streicher’s 1994 groupoid model and Voevodsky’s later contributions: the \emph{univalence axiom} (equating equivalent types) and \emph{higher inductive types}~\cite{hottbook} (defining spaces by specifying points and paths). Together, they enable HoTT to internalize homotopy-theoretic concepts directly into type theory. For computer scientists, this enriches program verification with geometric reasoning; for mathematicians, it offers a language where proofs are both human-readable and machine-checkable.
\section{Martin-\Lof's Dependent Type Theory}\label{sec:dtt}


% 前面还有一大堆

\begin{definition}
    Consider a type family $B$ over $A$.
    The \emph{dependent pair type} (or $\Sigma$-type) is defined to be the inductive type $\Sigma_{(x:A)} B(x)$ equipped with a pairing function
    \[
        (-,-) :\Prod_{(x:A)}  \Paren{B(x)\to\Sum_{(y:A)}B(y)}
    \]
\end{definition}
\section{Homotopical Explanation}\label{sec:homotopy}


\begin{table}[H]
    \centering
    \begin{tabular}{ll}
        \hline
        \emph{Type Theory} & \emph{Homotopy Theory} \\
        \hline
        Types              & Topological Spaces     \\
        Dependent Types    & Fibrations             \\
        Terms              & Points                 \\
        $\Sigma$ Type      & Total Space            \\
        Identity Type      & Path Fibration         \\
        Contractible Type  & Contractible Space     \\
        \hline
    \end{tabular}
    \caption{The homotopy interpretation~\cite{rijke2022introductionhomotopytypetheory}}\label{tab:homotopy-interp}
\end{table}

\begin{definition}
    Let $f,g:\prod_{(x:A)}P(x)$ be two dependent functions.
    The type of \emph{homotopies} from $f$ to $g$ is defined as
    \[
        f\sim g\Defeq \Prod_{(x:A)}f(x)=g(x)
    \]
\end{definition}

\begin{definition}
    A \emph{fiber bundle} structure on a space $E$, with fiber $F$, consists of a projection map $p: E\to B$ such that each point of $B$ has a neighborhood $U$ for which there is a
    homeomorphism $h:p^{-1}(U)\to U\times F$ making the following diagram commute
    \begin{figure}
        \begin{tikzcd}[row sep=large,column sep=huge]
            p^{-1}(U) \arrow[r, "h"] \arrow[dr, "p"'] & U\times F \arrow[d, "\pi_1"] \\
            & U
        \end{tikzcd}
    \end{figure}
\end{definition}

The fiber bundle structure is determined by the projection map $p:E\to B$,
but to indicate what the fiber is we sometimes write a fiber bundle as $F\to E\to B$,
a `short  exact sequence of spaces'.
The space $B$ is called the \emph{base space} of the bundle, and $E$ is the \emph{total space}.

% A map p : E  →  B is said to have the  homotopy lifting property with respect to a space X if, given a homotopy g  t  :X  →  B  and a map   ̃  g  0  :X  →  E lifting g  0  , so p   ̃  g  0  =g  0  , then there exists a homotopy   ̃  g  t  :X  →  E  lifting g  t  . From a formal point of view, this can be regarded as a special case of the  lift extension property for a pair (Z, A) , which asserts that every map Z  →  B has a  lift Z  →  E extending a given lift defined on the subspace A ⊂ Z . The case (Z, A) =  (X × I, X × {0}) is the homotopy lifting property.

\begin{definition}
    A \emph{fibration} is a map $p:E\to B$ having the homotopy lifting property with respect to all spaces $X$.
\end{definition}

\begin{definition}
    A \emph{contractible type} is a type which has, up to identification, only one term.
\end{definition}
\section{Conclusion}\label{sec:conc}

\bibliographystyle{amsplain}
\bibliography{refs}

\end{document}